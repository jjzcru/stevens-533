\section{Your boss listens nicely to your estimates, and tells you they are 20\% too large. How do you cut the effort by 20\% ?   What can you do about it?}
\noindent
We need to reduce the current hours of effort, 2127, by \textbf{20\%}, we need to do something that can save us around 425 hours of effort so our goal is to complete the project with 1702 hours of effort or less.  \newline\newline
\noindent
We can't change the actors because those are the minimum required to make the system work, we need the database and the inline system so the use cases we need to keep them as it is, and the technical factors are the ones that we need, our biggest weakness is the environmental factors.\newline \newline

\noindent
There is only one thing that we need to do to trigger a domino effect that will decrease effort and that is to change the programming language to one that everybody knows and this is what would happen:
\begin{enumerate}
    \item There is no benefit to using an obscure programming language, if we choose a language that everyone knows we can change the environmental requirement to 0. Just by doing that we reduce the effort to \textbf{1875}, that's a \textbf{12\%} decrease with just one change.
    \item If everyone knows the language that would increase the motivation in the team to work on the project because they do not need to learn a new language, let say that the motivation increase from a 2 to 4, that would give a total of \textbf{1774} that would be a \textbf{17\%} decrease from the original amount.
    \item But not only that since we are using a programming language that everyone knows that could increase the OO Programming Experience from a 1 to 3 that would give a total of \textbf{1673} that would be a total of \textbf{21\%}.
\end{enumerate}

\noindent
With that change we were able to reduce from \textbf{2127} hours of effort to \textbf{1673}, thats \textbf{454} hours of effort saved without changing anything in the specification or paying overtime.