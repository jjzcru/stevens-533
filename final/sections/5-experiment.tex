\section{Experiment Design}
There are two main things we want to know about the projects, to see if they are a good fit for a first-time contributor.

\subsection{Is the project alive?}
The first one is to figure out if a particular project is alive, by \textbf{alive} we mean a project that it's in active development and the community is active, there are examples of open source libraries that have an active community, the project still receive issues or pull request but from a roadmap perspective they already established that the project is done, so no new functionality will be added, an example of this is the library moment\cite{moment}.

For this reason, we are going to see how are the commits and issues behave over time to see if there is any trend, either an increase or decline, we are going to use that as an indicator of the "liveliness" of a project.

For the data that is going to be used we are going to group them by year, and group the projects together, and see what is the trendline for each of the metrics, these are:
\begin{enumerate}
    \fontsize{10pt}{10pt}
    \selectfont
    \item Trend of commits over the years
    \item Trend of opened issues over the years
    \item Trend of unique contributors per year
\end{enumerate}

\subsection{What is the overall sentiment of the project?}
The second one is related to the community behind these projects, we want to know if they are positive, negatives, or neutral, and to which degree.

For this we are going to use the comments inside the issues, the reason we are not using commit messages is because this one are related to the code, but comments in an issue have a more social nature, so comments would be a better indicator of the project “personality” that the commit message.

We fetch data from all the issues and each issue has a comment property that includes multiple messages sent by the contributor, collaborator, or any interested party, for each of these comments we performed a sentiment analysis on which we segmented the comment into words and group them into positive, neutral and negatives, then we see a comparative value between this and assign a score, if the word is positive we assign a positive score and if it’s negative we assign a negative value, we also keep a list of all the positive and negative words for future analysis.