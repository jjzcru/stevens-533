\section{Experiment Design}
There are two main things we want to know about the projects, to see if they are a good fit for a first-time contributor.

\subsection{Is the project alive?}
It is essential to determine whether or not the project is still being developed and maintained; by \textbf{alive}, we mean that there are still issues or pull requests being submitted to the project, but the project's roadmap has already established that no new functionality will be added; the library moment \cite{moment} is an example of this.

So, as an indicator of how active a project is, we will look at how commits and issues change over time; we will use this as a measure of its "liveliness."

For this analysis, we will group the data by year and then examine the trends for each of the following metrics:
\begin{enumerate}
    \fontsize{10pt}{10pt}
    \selectfont
    \item Trend of commits over the years
    \item Trend of opened issues over the years
    \item Trend of unique contributors per year
\end{enumerate}

\subsection{What is the overall sentiment of the project?}
Second, we would like to know whether the people working on these projects are positive, negative, or neutral, and to what extent.

Rather than using commit messages, which are more closely associated with the code, we will look to the comments within issues to get a sense of the project's "personality", because comments are more social.

There are multiple comments on each issue, and we perform a sentiment analysis on each of these comments; for each of these comments, we group the words into positive, neutral, and negative, then we see the comparative value between this and assign a score. If the word is positive, we assign a positive score; there are multiple comments for each word, and each comment has a sentiment analysis. If it is negative, we assign a negative value; we also keep a list of all positive and negative words for future reference. 

\pagebreak